\documentclass{article}
\usepackage[utf8]{inputenc}
\usepackage{amsmath}
\usepackage{fancyhdr}


\pagestyle{fancy}
\fancyhf{}
\fancyhead[L]{Ilias Dzhabbarov, SD-03}
\fancyhead[R]{31 August 2024}

\begin{document}

\section*{Week 1. Problem set}

\begin{enumerate}
    \item Which of the following $\lambda$-terms are \emph{closed} \cite[§5.1]{1}? Justify your answer.
    \begin{enumerate}
        \item $\lambda a.(\lambda a.b) \, a$
        \item $\lambda d.x \, (\lambda d.d)$
        \item $\lambda x.(\lambda x.x) \, x$
    \end{enumerate}

    \subsubsection*{Solution:}
    \begin{enumerate}
        \item let's start with inner term $(\lambda a.b) \, a$. we can assume that this is func that takes some 
        argument $a$ and retuns term $b$ from the outer scope. And this func immediatly called, so it is just $b$
        \begin{center}
            $(\lambda a.b) \, a = b$
        \end{center}
        so we have
        \begin{center}
            $\lambda a.(\lambda a.b) \, a \leftrightarrow \lambda a.b$
        \end{center}
        and obviously we have $b$ from outer scope
        \subsection*{Answer: Not closed}
        \item short explanation: $x$ is not an parameter of any lambda function, so it must be
        from outer scope
        \begin{center}
            $\lambda d.x \, (\lambda d.d)$
        \end{center}
        \subsection*{Answer: Not closed}
        \item By the same logic as in (a), we can assume
        \begin{center}
            $(\lambda x.x) \, x = x$
        \end{center} 
        so we have
        \begin{center}
            $\lambda x. x$
        \end{center} 
        No other variables, obviously closed
        \subsection*{Answer: Closed}
    \end{enumerate}


    
    \item Write down the \emph{call-by-value} evaluation sequence for the following $\lambda$-terms. Each step of the evaluation must correspond to a single $\beta$-reduction or an $\alpha$-conversion. You may introduce aliases for subterms.
    \subsubsection*{Solution:}
    \begin{enumerate}
        \item $(\lambda x.\lambda y.x) \, (\lambda z.y) \, (\lambda z.z) \, w$
        
        \begin{enumerate}
            \item $\beta$: $(\lambda x.\lambda \underline{y}.x) \, (\lambda z.y) \rightarrow $ $[x \mapsto (\lambda z.y)]\lambda \underline{y}.x$ $\rightarrow \lambda \underline{y}.\lambda z.y$\\
            underlined $\underline{y}$ and $z$ are variables that never used so let's rename them to the "\_" to keep things clear
            \item $\alpha$: $\lambda \underline{y}.\lambda z.y$ $\rightarrow$ $\lambda \_.\lambda \_.y$\\
            now we left with:
            \begin{center}
                $(\lambda \_.\lambda \_.y) \, (\lambda z.z) \, w$
            \end{center}
            so we just put two things in the function one after the other and obviously we will get $y$

            \item $\beta$: $(\lambda \_.\lambda \_.y) \, (\lambda z.z)$ $\rightarrow$ $...$ $\rightarrow$ $\lambda \_.y$ 
            \item $\beta$: $(\lambda \_.y) \, w$ $\rightarrow$ $...$  $\rightarrow$ $y$
            \subsection*{Answer: $y$}
        \end{enumerate}

        \item $(\lambda b.\lambda x.\lambda y.b \, y \, x)$ $(\lambda x.\lambda y.y)$
        \begin{enumerate}
            \item $\beta$: 
            $(\lambda b.\lambda x.\lambda y.b \, y \, x)$ $(\lambda x.\lambda y.y)$ 
            $\rightarrow$ $[b \mapsto (\lambda x.\lambda y.y)]$
            $\rightarrow$ $\lambda x.\lambda y.(\lambda x.\lambda y.y) \, y \, x$
            \item $\beta$ (innner-part): $(\lambda x.\lambda y.y) \, y \, x$
            $\rightarrow$ $[x \mapsto y](\lambda x.\lambda y.y) \, x$
            $\rightarrow$ $(\lambda y.y) \, x$
            \item $\beta$ $(\lambda y.y) \, x$ $\rightarrow$ $...$ $\rightarrow$ $x$
            \begin{center}
                $\lambda x.\lambda y. x$
            \end{center}
            \subsection*{Answer: $\lambda x.\lambda y. x$ (or $\texttt{tru}$)}
        \end{enumerate}        

    

        \item $(\lambda s.\lambda z.s \, (s \, z)) \, (\lambda b.\lambda x.\lambda y.b \, y \, x) \, (\lambda x.\lambda y.y)$
    \begin{enumerate}
        \item $\beta$: 
        $((\lambda s. \lambda z. s \, (s \, z)) \, (\lambda b.\lambda x.\lambda y.b \, y \, x)) \, (\lambda x.\lambda y.y)$
        $\rightarrow$ $[\lambda s \mapsto (\lambda b.\lambda x.\lambda y.b \, y \, x)]$
        $\rightarrow$ $(\lambda z. (\lambda b.\lambda x.\lambda y.b \, y \, x) \, ((\lambda b.\lambda x.\lambda y.b \, y \, x) \, z)) \, (\lambda x.\lambda y.y)$
        
        \item $\beta$:
        $(\lambda z. (\lambda b.\lambda x.\lambda y.b \, y \, x) \, (\lambda x.\lambda y.(z \, y \, x))) \, (\lambda x.\lambda y.y)$
        $\rightarrow$ $[z \mapsto (\lambda x.\lambda y.y)]$
        $\rightarrow$ $(\lambda b.\lambda x.\lambda y.b \, y \, x) \, (\lambda x.\lambda y.(\lambda x.\lambda y.y) \, y \, x)$
        
        \item $\alpha$: Renaming to avoid collision
        $\rightarrow$ $(\lambda b.\lambda x.\lambda y.b \, y \, x) \, (\lambda x_1.\lambda x_0.((z \, x_0) \, x_1))$
        
        \item $\beta$: 
        $(\lambda x.\lambda y.((\lambda x_1.\lambda x_0.(\lambda x.\lambda y.y \, x_0 \, x_1)) \, y \, x))$
        $\rightarrow$ $[\lambda x \mapsto y]$
        $\rightarrow$ $(\lambda x.\lambda y.((\lambda x_0.(z \, x_0 \, y)) \, x))$
        
        \item $\beta$:
        $(\lambda x.\lambda y.(z \, x \, y)) \, (\lambda x.\lambda y.y)$
        $\rightarrow$ $[\lambda z \mapsto (\lambda x.\lambda y.y)]$
        $\rightarrow$ $(\lambda x.\lambda y.y)$
        
        \item $\alpha$: Renaming for clarity
        $\rightarrow$ $(\lambda x.\lambda x_2.x_2)$
        
        \item $\beta$:
        $(\lambda x.\lambda y.((\lambda x.\lambda x_2.x_2) \, x \, y))$
        $\rightarrow$ $(\lambda x.\lambda y.(\lambda x_2.x_2 \, y))$
        $\rightarrow$ $(\lambda x.\lambda y.y)$
        
        \subsection*{Answer: $\lambda x.\lambda y.y$  (or $\texttt{fls}$)}
    \end{enumerate}

    \end{enumerate}
    
    
    \item Recall that with Church booleans \cite[§5.2]{1} we have the following encoding:
    \[
    \text{tru} = \lambda t.\lambda f.t
    \]
    \[
    \text{fls} = \lambda t.\lambda f.f
    \]
    \begin{enumerate}
        \item Using only bare $\lambda$-calculus (variables, $\lambda$-abstraction and application), write down a $\lambda$-term for logical equivalence (\texttt{eq}) of two Church booleans. You may \emph{not} use aliases.
        \item Verify your implementation of \texttt{eq} by writing down the evaluation sequence for the term \texttt{eq fls tru}. You must expand this term and then evaluate \emph{without} aliases.
        \subsubsection*{Solution:}
        \texttt{not} = $\lambda a. a $ \texttt{fls} \texttt{tru} = ($\lambda a. a $ $(\lambda t.\lambda f.f)$ $(\lambda t.\lambda f.t)$)\\
        \texttt{eq} = $\lambda p.\,\lambda q.\,p\,q$ (\texttt{not} $q$) =  $\lambda p.\,\lambda q.\,p\,q$ (($\lambda a. a $ $(\lambda t.\lambda f.f)$ $(\lambda t.\lambda f.t)$) $q$)
        \texttt{eq fls tru} = ($\lambda p.\,\lambda q.\,p\,q$ (($\lambda a. a $ $(\lambda t.\lambda f.f)$ $(\lambda t.\lambda f.t)$) $q$))
        $(\lambda t.\lambda f.f)$
        $(\lambda t.\lambda f.t)$

        Evaluation:

        $
        (((\lambda p. (\lambda q. ((p q) ((\lambda a. 
        ((a (\lambda t. (\lambda f. f))) 
        (\lambda t. (\lambda f. t)))) q)))) (\lambda t. (\lambda f. f))) (\lambda t. (\lambda f. t)))
        \\ \rightarrow (((\lambda p. (\lambda q. ((p q) ((q (\lambda t. (\lambda f. f))) (\lambda t. (\lambda f. t)))))) 
        (\lambda t. (\lambda f. f))) (\lambda t. (\lambda f. t)))
        \\ \rightarrow (((\lambda p. (\lambda q. ((p q) ((q (\lambda t. (\lambda f. f))) (\lambda t. (\lambda f. t)))))) 
        (\lambda t. (\lambda x_0. x_0))) (\lambda t. (\lambda f. t)))
        \\ \rightarrow ((\lambda q. (((\lambda t. (\lambda x_0. x_0)) q) ((q (\lambda t. (\lambda f. f))) (\lambda t. 
        (\lambda f. t))))) (\lambda t. (\lambda f. t)))
        \\ \rightarrow ((\lambda q. ((\lambda x_0. x_0) ((q (\lambda t. (\lambda f. f))) (\lambda t. (\lambda f. t))))) 
        (\lambda t. (\lambda f. t)))
        \\ \rightarrow ((\lambda q. ((q (\lambda t. (\lambda f. f))) (\lambda t. (\lambda f. t)))) 
        (\lambda t. (\lambda f. t)))
        \\ \rightarrow ((\lambda q. ((q (\lambda t. (\lambda f. f))) (\lambda t. (\lambda f. t)))) 
        (\lambda x_1. (\lambda f. x_1)))
        \\ \rightarrow (((\lambda x_1. (\lambda f. x_1)) (\lambda t. (\lambda f. f))) 
        (\lambda t. (\lambda f. t)))
        \\ \rightarrow ((\lambda f. (\lambda t. (\lambda f. f))) (\lambda t. (\lambda f. t)))
        \\ \rightarrow (\lambda t. (\lambda f. f))$ = \texttt{fls}
    \end{enumerate}
    \item Recall that with Church numerals...
    \begin{enumerate}
        \item write down a single $\lambda$-term for each of the following functions on natural numbers. You may not use aliases.
        \begin{enumerate}
            \item $n \mapsto 2n + 1$
            
            \texttt{inc} = $\lambda n. \lambda s. \lambda z. s(n(s)(z))$\\
            \texttt{times} = $\lambda a. \lambda b. a(\texttt{pls}(b))(\texttt{c0})$\\
            \texttt{pls} = $\lambda a. \lambda b. \lambda s. \lambda z. a(s)(b(s)(z))$\\

            $\lambda n :  \texttt{inc}(  \texttt{times}(n)(\texttt{c2}))$

            $\lambda n :  ($\lambda n. \lambda s. \lambda z. s(n(s)(z))$)
            (  $\lambda a. \lambda b. a($\lambda a. \lambda b. \lambda s. \lambda z. a(s)(b(s)(z))$(b))(\lambda t.\lambda f.f)$(n)(\lambda s.\lambda z.s (s z)))$
        
            \item $n \mapsto 2^{n+1}$\\
            
            c2 = $\lambda s.\lambda z.s (s z)$
            
            $\lambda n :  \texttt{c2(inc($n$))}$


        \end{enumerate}
    \end{enumerate}
\end{enumerate}


\end{document}
